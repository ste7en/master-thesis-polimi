\section{XR in Tourism}
\label{sec:background-tourism-xr}

%Over the years, XR has been used to offer new ways to interact with the surroundings in many different fields such as medicine, industry, rehabilitation, etc. %
The market for XR applications today includes growth sectors including health and military technology that have been growing for a while, compared to other types of applications that are only recently gaining ground \cite{honkanen_enhancing_2018}. The tourism industry is one of the latest commercial applications that offers incumbents new opportunities to expand their business and the XR companies they partner with to become tourism experts \cite{kwok_covid-19_2020}.
Until recently, cultural tourism represented little more than a niche market \cite{han_virtual_2019}, with the traveler's experience being limited to site-related activities only. It was predictable that it would become an important economic and social driver in Europe and worldwide as visitors' engagement and the learning experience increased. Han et al.~\cite{han_virtual_2019} argue how technology can open up new opportunities to reformulate the user experience. On the other hand, Bekele et al.~\cite{bekele_survey_2018} discuss the significant added value that cultural heritage sites and artifacts gain from the introduction of technologies such as AR and VR.

For what concerns the application of XR in tourism and Cultural Heritage (CH), there are multiple examples of them as evidenced in Bekele et al.~\cite{bekele_survey_2018} and Loureiro et al.~\cite{loureiro_20_2020}. Several areas, all of crucial importance, can be addressed by an innovation as the one mentioned above. Such technologies are perceived as a powerful educational but also entertainment tool, of high utility if one considers the possibility of language translation, navigation, booking and so on. The resulting benefits impact not only the final user but also local attractions and their business growth \cite{nayyar_virtual_2018}. 

\subsection{Acceptance of immersive technologies in tourism}
Many studies highlight the advantages of exploiting these technologies, considering their limitations and how much people are willing to switch from a more common type of experience to an XR one.
The studies conducted by Kim et al.~\cite{kim_empirical_2020} reveal empirical proofs in support of the last sentence. The model they propose shows that interactivity, experience-ability and amplified engagement are the three main strengths of AR and VR contents.  

Despite all the possible benefits, it is still necessary to consider also how people might react to XR. Some people might be interested or curious while others might prefer a traditional kind of trip. The feedback is usually different based on how accustomed users are to new technology: people more familiar with it are more willing to try and they are more concerned about the quality of the system. Those who have never had a chance to try it or have little to no knowledge are more concerned about the quality of the content \cite{bulchand-gidumal_tourists_2020}, \cite{pourfakhimi_acceptance_2020}.

Referring to the behavior of the user approaching a technology in a friendly manner is the acceptance whereas when referring to the adoption of a technology the focus is on the reasons for adopting it. The extensive research in progress by Pourfakhimi et al.~\cite{pourfakhimi_acceptance_2020} aims at exploring the thoughts, emotions, interactions and impacts of a user or a possible user who is motivated or willing to buy, use and interact with a technology, without omitting examples of tourism-related technologies such as VR.

During the so-called “decision phase” of a customer, it is determined whether or not the barriers of adopting a new technology will be broken through. According to Laurell et al.~\cite{laurell_exploring_2019}, in order to overcome these barriers and thus encourage the diffusion of technology itself, one has to rely both on its autonomous value and, above all, on the perceived value of the technology. This will have to be greater than the overall value related to the utility it brings to the individual (i.e. if it is easy or fun to use), the number of users already using it and the “availability of complementary goods of the old technology”. 

However, this is not the only limitation to be considered, since as with any technology in its infancy, XR is subject to some restrictions such as the duration of use, not to mention that it offers an experience which, even if it comes close to the real world, has a feeling of artificiality. Faced, though, with another type of constraints such as those imposed by the COVID-19 regulations, what might previously have been perceived almost as a threat becomes an asset \cite{guttentag_virtual_2020}. Virtual tours have experienced a surge since they were the only way to enjoy cultural experiences. An opportunity, if you can call it that, that has the responsibility to be up to the level of the original places being reconstructed in VR, as they might positively or negatively influence the user to visit a place in the near future. 

%Travel phases (qui o prima di TAM??)
\subsection{XR in Travel Industry}
What was seen in Bekele et al.~\cite{bekele_survey_2018} was then also found in other research works comparing the above results with those obtained in other applications and implementations, not only in a CH context but considering every step of the tourist journey. For example, Bulchand-Gidumal et al.~\cite{bulchand-gidumal_tourists_2020} divided the user journey in three main phases:
\begin{enumerate}
	\item the pre-travel where the user starts planning the trip, developing the expectation;
	\item the travel where the user experiences the planned trip;
	\item the post-travel where the user shares and re-experiences the journey.
\end{enumerate}

The research proves how AR and VR can be used in the different stages of travel, as a complement to and augmentation for a travel process or as a substitute to the tourism experience, allowing tourists to enjoy an experience without travelling. In the results of their study VR seems to be more appropriate for pre-travel and post-travel, while AR seems to be a good fit to use during the trip. VR in fact, can be used in the decision-making process, providing tourists a first glance of the destination and the potential activities. Moreover, VR can help planning in advance all the activities of a trip (e.g. getting information on the accessibility of a destination) or booking the accommodations and the activities themselves. This is thanks to key elements of VR (immersion, interactivity, presence) and the possibility to recreate, through 3D contents or 360$^{\circ}$ videos, a totally realistic environment. 
In the post-travel phase VR allows to share and enjoy 360$^{\circ}$ images and videos recorded during a trip, with a word-of-mouth effect on VR-enabled platforms.
In contrast to VR, where immersion in a virtual world is a desired feature, the goal of AR is to enhance the way in which the users perceive the real world, by adding layers of virtual objects. Therefore, they agree that Augmented Reality can provide tourists enhanced information on the sites, for example showing users how a certain site looked like at a different point in history, or a way to buy product and services while visiting a destination.

In Egger et al.~\cite{egger_augmented_2020} a comprehensive description of how AR and VR can be implemented in the tourism marketing is given. Before booking a travel AR can augment brochures, flyers, websites and flight tickets to boost the storytelling for a brand, while VR is essential to preview parts of a journey in a “try before you buy” experience. During the trip, POIs can be suggested by an AR application and CH sites are enhanced with additional information or narratives, adding a gamification and storytelling layer. On the other hand, virtual walks substitute inaccessible locations or not existent sites.

Some concrete examples of XR experiences in the industry are given by Jenny \cite{jenny_enhancing_2017}. The Finnish airline Finnair created an immersive and interactive VR visit of their new air fleet to showcase the wide seats, large windows and ambient lights; the shipping operator Finnlines, instead, used 3D photo tours to advertise their renovated vessels for ferry trips. The hotel chain Marriott in cooperation with Samsung created the \textit{VRoom Service} for their customers, giving the opportunity to visit multiple countries from their rooms and order food and beverages to the room service. The same company uses VR to provide virtual tours of their accommodations and promote the booking process \cite{adachi_using_2020}; the Hub Hotel in England used AR to render maps of the city on the walls of the rooms, while through the Holiday Inn experience it is possible to see virtual images of famous people at various points \cite{ercan_examination_2020}.

Other researchers like Adachi et al.~\cite{adachi_using_2020} studied and confirmed the effectiveness of VR headset among different media types, stating that an \gls{HMD} acts on the self-presence feeling of a user, increasing their desire for a destination, as also confirmed by Kang \cite{kang_impact_2020} and Leung et al. for hotel VR commercials \cite{leung_fad_2020}.

\subsection{XR for Cultural Heritage}
% Altri titoli:
% Review of previous XR applications in cultural tourism
% A review of XR (applications) for Cultural Heritage
% examples of MR apps...
Bekele et al.~\cite{bekele_survey_2018} examined augmented, virtual, and mixed reality from a cultural heritage perspective. They focused on technical aspects such as tracking and registration, VE modeling, presentation, tracking, input devices, interaction interfaces and systems. Notwithstanding them, their overview likewise breaks down the different application zones for XR in CH, namely: education, exhibition enhancement, exploration, reconstruction, and virtual museums.

The creators propose that a definitive decision of enabling technology should rely upon the experience that an application is expected to give. Even though augmented, virtual and mixed reality can be used interchangeably to achieve same purposes, their results show that AR is preferable for exhibition enhancement. Similarly, VR seems better for virtual museums, and mixed reality most viable for both indoor and outdoor reconstruction applications.

\subsubsection{Indoor and outdoor XR applications}

The spatial setting for exhibiting virtual and augmented representation is binding. The design for environments such as museums, galleries and heritage sites hinges on the power to build and tell stories that settle in mind and become knowledge. The visitor must feel involved, not only with their body, but also by moving and interacting with the scene, in order to overturn the figure of the passive observer. Pedersen et al.~\cite{pedersen_more_2017} explain that holographic reality with a 3D point of view is able to impress a feeling of depth and solidity that is quite close to reality. Sometimes the actual content of the museum itself becomes the author of the narrative, immersing the visitor in the museum scene and tour. Once the points of interest have been set, a visit route is designed which takes visitors back to the past. This becomes a great opportunity for locations such as museums, which take on a new role by merging the value of heritage with technological elements that bring innovation, learning through experience and entertainment. The model developed by Trunfio and Campana \cite{jung_augmented_2020} brings the interaction between the different actors involved such as the visitors, the services offered by the museum and the employment of new technologies such as mixed reality. Investing in museums also means contributing to the preservation of cultural and artistic heritage, improving the visibility of the museum and defining new business scenarios.
To experience mixed reality indoors, sometimes you don't even have to travel far. That is the case with the Ascape application that allows to be immersed in the experience of a real tourist while staying at home \cite{cranmer_understanding_2016}. Thanks to the use of 360$^{\circ}$ videos it is possible to explore places from different perspectives. When it comes, instead, to influencing users to visit through an internal preview of the location, it is preferable to adopt \gls{HMD}, the customer has the feeling of “being there” and has a definite intention to visit that place.

%outdoor
Outdoor applications are more suitable for a reconstruction approach, because it is then possible to overlay reconstructed historical views over the real world in a “real-virtual” representation. Complementary, two forms of blended 3D assets are needed when reconstructing cultural heritage in a MR application: present and past views \cite{bekele_survey_2018}. Moreover, a reconstruction theme is often applied to outdoor sites that have been demolished or worn away. When it is not synonymous with extension, it is known as “diminished reality”: virtual elements are removed to restore a scenario to its original state of appearance \cite{honkanen_enhancing_2018}. 
The majority of applications serve virtual museums, followed by education, reconstruction, and exploration purposes.
%HMD, desktop, and CAVE displays are the common choices for displaying a virtual environment. 
In some cases it is convenient to think of outdoor locations, even very wide ones, as museum exhibits. Outdoor environments are more dispersive, but as mentioned before, one of the goals of mixed reality applications is exploration. Therefore it might be relevant to guide users towards the place they are looking for \cite{jenny_enhancing_2017} or to drive them to discover sites they have not visited yet. Exploration is more related to tangible resources of cultural heritage, while reconstruction also uses intangible resources or both, combined to overlay augmented and virtual elements on top of reality \cite{bekele_survey_2018}. According to the location's requirements, it can be decided whether to use the involved technology aiming at replacing reality or at being complementary to it \cite{racz_virtual_2019}. In the first case, it may be a real constraint due to the inaccessibility of some places that are moved into a virtual world to be available. This is also referred to as historical-cultural events. In the latter case, it can be an entertainment and promotional tool for the site to arouse the interest of visitors.
Some examples of AR experiences applied to urban or touristic-cultural contexts demonstrate the superiority of outdoor realism and involvement over indoor, precisely because outdoors favors the use of \gls{HMD} cameras and therefore of AR/VR content \cite{iacoviello_holocities_2020}. Visiting a city is no longer a one-to-one relationship with the mobile application, but it becomes a co-participation, it changes the point of view of the observer visiting that place and becomes an experience that the user shares with nature and art and sometimes with other users.
If indoor cultural tourism exploiting mixed reality has some limitations such as the field of view of \glspl{HMD} that forces the user to always stand at a certain distance from objects thus avoiding that they appear larger than expected, outdoor tourism is even more insidious. A big and still open challenge, in indeed, is the excessive natural light that could limit the view through the viewers and the instability of the building recognizer that allows the virtual reconstruction \cite{debandi_enhancing_2018}. Furthermore, outdoor tracking is always less accurate than indoor tracking. MAR applications need to consider that objects might move slightly or flutter. Building scenarios with AR content requires a great deal of effort, due to the need to customize 3D objects for the specific outdoor space. This implies two challenges identified by Xu et al.~\cite{xu_exploring_2018}, one related to the costs that become demanding and the other related to the immersion in an external environment. The risk, in fact, is to bump into real objects because of the feeling of being completely absorbed in the AR experience.

%distinction physical-remote (?)
So far visitors or users have been identified without specifying whether or not they were physically present at the site of interest \cite{gaberli_tourism_2019}. The new digital era has made it possible for everyone to be a citizen, or rather a tourist, of the world. Being able to visit and learn about places remotely, through the use of the Internet, initially broadened the horizons for all those who could not physically travel. Subsequently, it became a tool for exploring a destination even before arriving there. With the introduction of XR technologies, a new shape of tourism has spread, which has also become very successful from an entrepreneurial point of view. Remote tourism has become a resource for those who intend to travel and those who unfortunately cannot. The remote experience has reached significant levels of popularity, travel communities are a source of information and the number of travel video log views exceeds one billion \cite{adachi_using_2020}. Despite this, there is no comparison between a computer display showing a 360$^{\circ}$ video and the feeling of real presence offered by an \gls{HMD}. Also from a promotional point of view, the headset has a superior persuasive effect due to the quality of the social presence that provides the virtual and augmented environment with an appearance of humanity.
%Museum categorization/virtual museum 
Online museum experiences are explored in depth by studies conducted by Doukianou et al.~\cite{doukianou_beyond_2020} revealing, that they are not very popular, so they should be defined as almost obsolete tools. Virtual museums are presented as a brochure version for promotional purposes and as a collection of contents for educational purposes. The aim is to encourage visitors to come back and repeat the museum tour. It becomes very difficult to invest in such visits if the user experience is not appropriate. In some cases virtual museums include serious games that try to tell a story and increase the value of the experience. %In spite of this, the percentage of visitors who say that museums in their proximity are not embracing technological and innovative solutions is close to 36\%, which says a lot about the interaction of users with such technologies.

\subsubsection{Examples}

There are a lot of examples of AR applications in tourism and cultural tourism. The first MAR app specific for the tourism industry was \textit{Tuscany+}, published in 2010, a digital tourist guide in different languages offering information about sightseeing, accommodations, restaurants and other point of interest for travelers \cite{kounavis_enhancing_2012}, \cite{nayyar_virtual_2018}. Similar applications emerged in the following years: the \textit{StreetMuseum} app, developed for the Museum of London, allows to see historical pictures and additional information of specific places by pointing the smartphone towards streets and buildings. Other MAR apps also offer adventures for users as for the case of \textit{Urban Sleuth} where a multiplayer VE is mixed with the real one in order to solve mysteries by collecting hints exploring a city, its neighborhoods and historical sights \cite{kounavis_enhancing_2012}. A similarity between this kind of apps and the famous AR game Pokémon Go has been found by Nayyar et al.~\cite{nayyar_virtual_2018} where the authors believe that strategically placed “PokeStops” in relevant sights can strengthen the overall tourist experience, as implemented in \textit{Get to Know Moscow} \cite{honkanen_enhancing_2018} where the user is looking for historical characters.
% Gaming experience
Different examples in the literature show the adoption and effectiveness of a gamified AR experience set in historical attractions. Honkanen's Master Thesis \cite{honkanen_enhancing_2018} suggests and implements a double user experience solution in a cultural heritage application: a gamified version for competitive users, and a standard version for non-competitive ones. The former enhances the attraction through an AR storytelling taking place in the past and with the users trying to solve some questions. Users who win the game have the chance to show their name on the leaderboard. Linaza et al.~\cite{linaza_interactive_2007} prototype and test an AR application based on a board game enhanced by augmented cards and virtual avatars guiding the user throughout historical events; their article also provides a useful and detailed reference to design an interactive experience on a historical event based on well-known game dynamics.

Some authors have approached and analyzed the phenomenon of tourists' experience enhancement through MAR. Mobile AR apps act as a digital tourist guide “on-demand”, displaying content upon request and by travelling around the city \cite{kounavis_enhancing_2012}, adding \textit{"a layer of guidance, content and entertainment to physical locations seen through the smartphone's camera view"} \cite{nayyar_virtual_2018}, thus minimizing the effect of information overload \cite{kounavis_enhancing_2012} because in AR information are organized in layers that can be hidden or shown. Lando \cite{lando_how_2017} explores how museum curators themselves can complement exhibitions through design options and frameworks aimed at developing effective AR learning experiences. An interesting framework described by the author is the CHESS framework that proposes to create a story-driven cultural exploration of museums via a plot and scripted paths augmented by multimedia resources as audio, videos and more.

AR is increasingly proposed for \gls{HMD} technology that fits over the eyes. Holographic AR, or Mixed Reality, as a matter of fact, integrates a more immersive AR experience because it involves using a headset that provides a 3D viewpoint.
Mixed Reality applications in tourism have shown the potentials of this technology in creating unforgettable and special experiences, as described by Gaberli \cite{gaberli_tourism_2019}. In his work it is reported how a conceptual model for the UX of a MR app in the Ara Pacis Museum \cite{jung_impact_2020} in Rome can proclaim \textit{“the role of the museum as an immersive cultural site”}; the same positive effects are evident when the intention to visit a cultural site in Naples increase after a VR experience with a wearable headset \cite{marasco_exploring_2018}.
% HMD
Microsoft HoloLens 2 and Meta 2 are currently two of the most recognized hardware platforms for Mixed Reality and holographic computing, creating holograms that appear extremely realistic. Further, gestural interaction provides users with a more embodied and intuitive experience. 
Iacoviello and Zappia \cite{iacoviello_holocities_2020} described their \textit{HoloCities} application for groups of tourists discovering a city using Microsoft HoloLens and interacting by means of gestures and voice commands. Participation and collaboration is promoted by the use of the app and the discovery of the visited place is enhanced by multimedia data stored in the archives of the Italian public broadcasting company RAI. Objects like monuments and buildings are detected and the corresponding hologram is rendered in the scene, allowing users to interact with them and encouraging cultural discussions. Furthermore, holograms can be manipulated (translation, rotation, scaling) and enriched with detailed information or videos.
The authors of the paper also studied and analyzed the different menus (either 2D or 3D) and UI components, whose results in their user test show a perceived easiness of use and intuitiveness by users.

Other examples of MR applications for Cultural Heritage are given by Pedersen et al.~\cite{pedersen_more_2017} and Debandi et al.~\cite{debandi_enhancing_2018}.
The former is a MR application for Meta 2 called \textit{Tombseer}. It is a CH application that uses MR to enhance an Egyptian Tomb. It aims to revitalize an exhibit by making virtual representations of Egyptian artifacts seem tangible; in this way, users can learn and have fun also in an empty room.
The latter presents a MR application for outdoor, enhancing the user experience during cultural tours. Virtual contents are anchored to some positions in the real space and the user can interact with them. The application gathers images of what the user is watching several times per minute; images are sent to the cloud architecture where a visual search engine is running. In this way, objects of interest framed by the user can be identified and a notification about a recognized object is sent back to the user, who can receive augmented contents about the object. 
Both the applications can display textual information and movies about the framed monument/building/artwork and a semi transparent silhouette of the recognized item can be used to align the user with respect to the target, thus enhancing the tracking robustness. When the AR application tracks the object, 3D virtual assets can be overlapped to the target. What changes between the two is the environment for which the application was designed: the first for an indoor space while the second for an outdoor space.

% VR
Entirely virtual contents are generated in VE, most used for proposes such as virtual museums \cite{doukianou_beyond_2020}, virtual reconstructions, explorations and education \cite{bekele_survey_2018}. Researchers have exploited, in their works, advanced techniques like photogrammetry and laser scanning to acquire 3D models of cultural heritage sites and museums' exhibits in order to allow tourists to visit inaccessible sites virtually or explore attractions from different locations. 
For instance, a Finnish chocolate industry provided VR headsets to visitors in its factories to enter in the production lines and see how their products are made, which beforehand was not possible due to hygiene regulations. UNICEF, instead, rendered in VR a Syrian refugee camp in Jordan through a 360$^{\circ}$ video where a Syrian child guides participants in a tour of her village, in order to raise awareness on refugees' conditions.
Other kinds of VR applications in CH are narrative games in virtual museums, as reported by Doukianou et al.~\cite{doukianou_beyond_2020} who believe that games \textit{“can create the narrative that is missing from the museums that unfolds a story while people are playing"}.

%\subsubsection{Indoor and outdoor XR applications}
%indoor
%In general, indoor applications are more suited to exhibition enhancement experiences since physical museums tend to use such applications for virtual tour guidance more often than outdoor CHs. 


% Il pezzo seguente è da spostare (o integrare) da qualche parte
%Device-based, sensor-based, and hybrid interfaces are used most often to interact with the virtual environment, though multimodal interfaces are more intuitive and natural. The most common VR systems employed in cultural heritage are semi-immersive and fully immersive. Recent advances in HMD, tracking sensors, and computer graphics technologies enable very realistic modelling and real-time rendering of virtual environments, but this has yet to be widely adopted in CH. Most of the MR applications are designed for non-immersive experiences, instead. Hybrid tracking, often a fusion of GPS and marker-less, are used. Mobile displays are commonly used to present visual and audio content. However, some systems also use custom-built \glspl{HMD}, and combinations of different types of presentation devices to display real-virtual content. MR applications in the CH domain are not as widespread as AR and VR. This is understandable given that the technological aspects of MR are more complex and many of them are still in their infancy.

%Social
\subsection{Future Trends}
\label{subsec:background-xr-future}
As XR is a continually evolving technology, its applications also change over time, applying new techniques and improvements to make the experience better and feel more natural. In the following sections we selected and analyzed some current and future topics that we believe will play a decisive role in the evolution and adoption of XR technology.
\subsubsection{5G}
The development of 5G technology has led to the achievement of faster data distribution and reduced latency so that content can be accessed more easily. As Patil et al.~\cite{patil_accelerating_2019} claim, receiving data faster also means having an application that requires less time to react to the user’s request, displaying up-to-date data almost immediately with the possibility, for the XR authors, to upload their improved contents for a richer experience. Debandi et al.~\cite{debandi_enhancing_2018} showcased, in their paper, how a 5G connection can be useful to exploit the low latency connection to a visual search engine and later retrieve AR and media contents to enhance information about historical buildings, monuments and artworks.
Other authors, furthermore, believe that the fifth generation connectivity will democratize the computational power needed for XR through edge computing servers close to the users. Doukianou et al.~\cite{doukianou_beyond_2020}, indeed, suggest that the shift to the cloud will release devices from high computational tasks and allow contents to be accessed by more users with different devices without compatibility requirements.

\subsubsection{Social Interactions}
One possible application of 5G could be a collaborative environment where any interaction can be immediately reflected on all the participants with little to no latency between cause and effect. In Iacoviello and Zappia \cite{iacoviello_holocities_2020} this application is what they refer to as \textit{Shared Reality}, where a guide, being them present in the room or in remote, manipulates 3D models whose rendering is shared among the users. In a second scenario, without a guide, the network is the means through which shared information and interactions are shown to a group of tourists thus promoting participation and collaboration between them. The \textit{Shared Reality} concept is an extension of the collaborative interfaces described by Bekele et al.~\cite{bekele_survey_2018} in their analysis of the six types of interfaces for augmented, virtual and mixed reality. A collaborative interface made possible by wearing a see-through \gls{HMD} enable tourists a collaborative navigation and information retrieval at cultural heritage sites, with additional functionalities to follow other users or guide them throughout their visit.

A similar experience is found by Kersten et al.~\cite{kersten_virtual_2018}, where a VR environment enables multiple users to visit 3D-generated cultural heritage sites regardless of their location in the real world, while the MAR counterpart is suggested by Kounavis et al.~\cite{kounavis_enhancing_2012} that highlight the uniqueness of mobile devices in enhancing social interaction among users.

Nowadays, the idea of sharing, collaboration and networking goes hand in hand with the meaning of social. It did not take long for MR to become a marketing tool, a powerful means of information distribution. Given their immense use today, social platforms have become the perfect market to be invested in. The creators of \textit{Geollery} didn't miss this opportunity, they built a system that "renders a mirrored world in real time" by displaying visible advertisements and much more in MR \cite{du_experiencing_2019}. It allows users to share their contents and interact with other users through geotags. Potential applications for crowdsourcing tourism are also discussed. Social media thus gains a spatial context as a 3D environment where social interaction acquires a whole new depth. 

\subsubsection{Machine Learning}
Gathering and analyzing people's behavior while they are using XR devices to see how they react to the content proposed and see what the effects of the media are might be useful to improve the experiences offered. In this case, it could be beneficial to analyze other kinds of information, such as biophysical data \cite{marchiori_measuring_2017} obtained through sensors on the used device. All the relevant statistics collected can then be passed to Machine Learning algorithms to extrapolate useful patterns that can then be used to design more engaging content \cite{laurell_exploring_2019}.

Algorithms such as partial least squares (PLS) regression analysis were used to test a research model centered on social presence in terms of "being in an environment" \cite{jung_effects_2016}. One of the objectives of those who invest in a new technological dimension is to engage customers, i.e. mostly visitors, and to provide them with an experience that makes them want to come back. Beyond the quality of the product or service, it is essential to increase the social presence, to give the feeling of really living that reality in favor of the whole tourist visit. 

\subsubsection{Virtual Agents}
The whole XR experience can be changed by introducing virtual agents, an avatar that can interact with users at any time, providing suggestions and information by communicating with them. As AI improves, these companions will become smarter, giving them the possibility to be aware of the context and surrounding area so that they can react accordingly to the proposed situation \cite{norouzi_systematic_2020} and help tourists in their decision-making process, increasing their satisfaction and willingness to visit back \cite{loureiro_20_2020}.
Doukianou et al.~\cite{doukianou_beyond_2020} report the use of an embodied conversational agent that conveys a feeling of empathy while asking questions to test the knowledge of users after having conducted them through the exhibition. The same kind of application of a virtual agent is presented by Linaza et al.~\cite{linaza_interactive_2007} where a virtual avatar serves as a personal guide in a historical interactive game.

\subsubsection{Social distancing}
Another aspect that must be taken into account and belongs to the present more than to the future is the situation due to the COVID-19 pandemic. Tourism has shown its vulnerability to the restrictions that have been imposed to decrease the diffusion of the virus. In this situation, XR might have a significant role as it can provide experiences while respecting both the social distancing rules and all the additional measures that have been taken by the various governments. This different approach might help at recovering in the current situation while also boosting the integration of XR as part of tourism \cite{kwok_covid-19_2020}.

Atsiz \cite{atsiz_virtual_2021} focused their attention also to the post COVID-19 period, suggesting the adaption - by museums - of their exhibits into VR in order to motivate tourists during their decision making process. It is also reported the successful promotion of VR by the British Museum and the Göbeklitepe Archaeological Site in Turkey during the COVID-19 lockdown, with many people virtually visiting these cultural attractions.
\\

We have seen how these technologies can bring benefits to the tourism and cultural heritage sectors but what changes as well is the design of user experiences and human-machine interactions in each of them. This is not only due to the different balance and characteristics of the virtual and real elements present, but also to the enabling technology used to live the experience.