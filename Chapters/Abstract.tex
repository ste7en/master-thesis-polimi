\chapter{Abstract}

Nowadays, existing approaches for creating Extended Reality (XR) experiences mostly entail the use of Integrated Development Environments (IDEs) or software libraries (APIs), while the adoption of a high-level approach -- allowing to abstract from the implementation -- is however confined to technology-dependent methods or close descendants of standard meta-models.
What is needed, therefore, is a universal model that allows XR applications to be designed at the conceptual level, without relying on solutions that are domain-specific or that exclude non-programmers from designing. This thesis presents the XRM Model, a theoretical approach oriented towards the conceptualisation of XR experiences, developed from a comparative study of existing models in the literature. The XRM Model is adopted for the implementation of a High-Level Editor to support XR application designers, which was validated in a usability test with users based on the ISO 9241-11 standard. In conclusion, an experience designed for an archaeological complex was considered as a case study in order to show the benefits resulting from the adoption of the XRM and the ART Editor.


\paragraph{Keywords}  Human-Computer Interaction; Interaction Design; Extended Reality; Conceptual Modelling; Authoring tools; Tourism; Cultural Heritage