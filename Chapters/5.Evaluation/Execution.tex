\section{Study Execution}
\label{sec:evaluation-execution}

The usability evaluation test -- being it remote and unmoderated -- is asynchronous, allowing each participant to decide which moment is best to accomplish it, requiring between 30 and 40 minutes; the execution of the study is divided into the following phases:
\begin{enumerate}
    \item \textbf{Introduction phase.} The introduction starts with a welcome message, introducing the topic of the test and what it is about. It then reminds the participants that, in order to correctly execute the Loop11 platform and the browser extension, they need to run it on a desktop PC using Google Chrome or Mozilla Firefox and to allow the screen recording permission request. Later, it is required to watch a video tutorial (\url{https://youtu.be/4jxPWBWVn0M}) of the duration of three minutes to get a better understanding of the tool; in conclusion, it is stated that by clicking on the button to start the test the participants give their consent to share the data about the usage of the website, such as the task completion time, screen recording and answers to the questionnaires.
    \item \textbf{Task execution phase.} This phase takes place on the Loop11 platform, where the ART Editor website is embedded, in order to run JavaScript code that will record the screen and present a ’Show’ button to show:
        \begin{enumerate}
            \item the task description
            \item "Abandon the test" button, to leave the test
            \item Mark the task as complete and pass to the next one" button to go on in the test.
        \end{enumerate}
    Each task is bound to a time limit of 10 minutes after which the platform loads the next task or the next questionnaire. During this phase it is recorded the screen of the device used to carry out the experiment.
    \item \textbf{Questionnaire phase.} In this phase the user answers the \gls{SUS} Questionnaire, three qualitative and open-ended questions on their experience throughout the test, plus user demographic data for reporting purposes (i.e., age and sex), in addition to close-ended user profile validation questions (\autoref{appendix:openended}).
\end{enumerate}