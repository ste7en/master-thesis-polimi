\chapter{ART Editor Usability Evaluation}
\label{ch:evaluation}

The evaluation of the ART Editor aims at testing its usability according to measures defined in the standard ISO 9241-11 \cite{iso20189241} -- namely \emph{effectiveness}, \emph{efficiency} and \emph{satisfaction} -- by users to achieve the test goals. While the data analysis metrics will be described in \autoref{sec:evaluation-approach}, along with the recruitment of participants and the kind of test performed, a clear definition of these measures is given:
\begin{definition}[Effectiveness]
\label{def:effectiveness}
A measure of the accuracy and completeness with which users achieve specified goals.
\end{definition}

\begin{definition}[Efficiency]
A measure of the resources expended in relation to the accuracy and completeness with which users achieve goals.
\end{definition}

\begin{definition}[Satisfaction]
The freedom from discomfort, and positive attitudes towards the use of the product.
\end{definition}

Goal of the test is to prove that the Editor has been correctly designed to provide participants a useful and user-friendly tool to develop \gls{XR} experiences without difficulties. At the end of the test, we expect users to have accomplished the following goals:
\begin{itemize}
    \item[\textbf{G1}] Users familiarize with the interface, its menu and interactions.
    \item[\textbf{G2}] Users are able to manipulate a scene design by modeling a simple scenario.
    \item[\textbf{G3}] Users are able to design multiple scenes in the application starting from a textual description.
    \item[\textbf{G4}] Users know how to customize the high-level description of a scene, adding labels, tags and secondary information.
    \item[\textbf{G5}] Users are able to export an already designed experience.
\end{itemize}

In order to provide accurate results in accordance to the final users and stakeholders of the ART Framework, we identified two participant profiles that could fit as users in our study:
\begin{itemize}
    \item[\textbf{P1}] Users without any experience in \gls{AR}/\gls{VR} development neither in computer science. This type of users reflect the role of art curators in a touristic domain and are useful to test the easiness and level of abstraction from technical details usually involved in the development of these applications.
    \item[\textbf{P2}] Users with experience in \gls{AR}/\gls{VR} development and with a computer science background. This type of users do not reflect the end user of the framework; however, we included them not only for easiness of finding this kind of participants in a technical university, but also because they could validate the rightness of application interface abstraction and concept abstraction of the designed solution.
\end{itemize}

\section{Usability Evaluation Approach}
\label{sec:evaluation-approach}

To evaluate the ART Editor a task-based test modality has been chosen, composed by four different tasks. All the tasks involve the design, through the Editor, of specific experiences; during the experiment the following metrics are collected:
\begin{itemize}
    \item[-]The task completion rate, measure of the effectiveness.
    \item[-]The average time required to complete a task, measure of the efficiency.
\end{itemize}

Satisfaction is then measured by means of quantitative and qualitative data collected at the end of all the tasks; the quantitative measure of satisfaction is guaranteed by the System Usability Score \cite{brooke_sus_1996} (in \autoref{appendix:sus}), while a qualitative indication of satisfaction is given by an open-ended questionnaire (\autoref{appendix:openended}).

The method we chose to perform the Usability Evaluation experiment and collect all the relevant data is the unmoderated remote test, consisting in the execution through an external platform (Loop11\footnote{\url{https://www.loop11.com/}}) of a recorded session in users' browsers in which a textual description precedes each task and an always on-top button (\autoref{fig:loop-11-button}) allows to either read again the instructions, end the task or abandon the test (\autoref{fig:loop-11-popup}).

\begin{figure}[H]
    \begin{subfigure}{\textwidth}
    \centering
    \includegraphics[width=\linewidth]{Figures/Evaluation/eval-platform1.png}
    \caption{}
    \label{fig:loop-11-popup}
    \end{subfigure}
    \begin{subfigure}{\textwidth}
    \centering
    \includegraphics[width=\linewidth]{Figures/Evaluation/eval-platform2.png}
    \caption{}
    \label{fig:loop-11-button}
    \end{subfigure}
\caption{The Loop11 platform executing a test session.}
\label{fig:loop-11}
\end{figure}
\section{Scenarios and Tasks}
\label{sec:evaluation-scenarios}

The test is composed by four tasks: each task has multiple goals defined as 'task rationale'; the first two tasks start from an empty project and require users to model the described experience from scratch, the last two tasks instead are presented with an already-modeled experience in which users are required to perform some changes or complete them to have a correctly designed experience.
The list of task rationales R1--R12 is described below and \autoref{table:evaluation-task-rationale} shows which task is based on these objectives.
\begin{itemize}
    \item[R1]Add multiple state and action nodes
    \item[R2]Add action and state tags
    \item[R3]Link states to actions and vice versa
    \item[R4]Keep coherent identifiers when referring to the same object
    \item[R5]Change tag properties
    \item[R6]Edit labels
    \item[R7]Add two conditional action nodes
    \item[R8]Link back to an initial state
    \item[R9]Add more than one object to the same node
    \item[R10]Add more than one tag to the same object
    \item[R11]Edit the recipe name
    \item[R12]Save the modeled experience
\end{itemize}

\begin{table}[h!]
\centering
\begin{tabular}{|l|c|c|c|c|c|c|c|c|c|c|c|c|} 
\cline{2-13}
\multicolumn{1}{l|}{} & \multicolumn{1}{l|}{R1} & \multicolumn{1}{l|}{R2} & \multicolumn{1}{l|}{R3} & \multicolumn{1}{l|}{R4} & \multicolumn{1}{l|}{R5} & \multicolumn{1}{l|}{R6} & \multicolumn{1}{l|}{R7} & \multicolumn{1}{l|}{R8} & \multicolumn{1}{l|}{R9} & \multicolumn{1}{l|}{R10} & \multicolumn{1}{l|}{R11} & \multicolumn{1}{l|}{R12}  \\ 
\hline
\textit{Task 1}       & x & x & x & x &   &   &   &   &   &    &    &     \\ 
\hline
\textit{Task 2}       & x & x & x & x & x & x &   &   &   &    &    &     \\ 
\hline
\textit{Task 3}       &   & x &   & x & x & x & x & x &   &    &    &     \\ 
\hline
\textit{Task 4}       &   &   &   &   &   &   &   &   & x & x  & x  & x   \\
\hline
\end{tabular}
\caption{Task rationales by task.}
\label{table:evaluation-task-rationale}
\end{table}


\subsection*{Task 1}
    \textit{Instructions:} 
\begin{enumerate}
    \item Given the initial \textit{State Node} in the diagram, place a 3D Model inside it, chosen from the list of Elements on the left.
        \begin{itemize}
        \item[-] add a \textit{State Tag} to make it \textbf{Visible}.
        \end{itemize}
    \item Insert an \textit{Action Node} containing the 3D Model element and the \textit{Action Tag} \textbf{Tap} on the 3D Model.
    \item Connect the previously made \textit{State Node} to the \textit{Action Node}.
    \item Add a new \textit{State Node} and link the last \textit{Action Node} to it. Within this new \textit{State Node} add a 2D Text and add a \textit{State Tag} to make it \textbf{Visible}.
    \item Change the ID of the 3D Model editing the little dark blue square containing the ID number to make it equal to the previous ID of the 3D Model.
\end{enumerate}
The solution to this task is shown in \autoref{fig:task1-sol}.
\begin{figure}[h]
    \centering
    \includegraphics[width=\linewidth]{Figures/Evaluation/Tasks/task1-sol.png}
    \caption{Solution to task 1}
    \label{fig:task1-sol}
\end{figure}

\subsection*{Task 2}
\textit{Instructions:} 
\begin{enumerate}
    \item Given the initial \textit{State Node} in the diagram, place a 2D Video inside it
        \begin{itemize}
            \item[-] add the \textit{State Tag} \textbf{Visible} to it.
        \end{itemize}
    \item Add an \textit{Action Node} (with the 2D Video inside) linked to the last created \textit{State Node} and model its final \textit{State Node} as: when the user \textbf{Taps} on the video it becomes \textbf{On Play}.
    \item Change the 2D Video Visible \textit{State Tag} options in:
        \begin{itemize}
            \item[-] 3 seconds fade in animation
            \item[-] after 2 seconds.
        \end{itemize}
    \item Change the \textit{Label Name} in “First Recording”.
    \item Change the 2D Video \textbf{On Play} \textit{State Tag} options in: Loop enabled.
    \item Sync IDs and \textit{Label Name}, they must be coherent in the whole sequence.
\end{enumerate}
The solution to this task is shown in \autoref{fig:task2-sol}.
\begin{figure}[h]
    \centering
    \includegraphics[width=\linewidth]{Figures/Evaluation/Tasks/task2-sol.png}
    \caption{Solution to task 2}
    \label{fig:task2-sol}
\end{figure}

\subsection*{Task 3}
\textit{Instructions:} 
The experience already modeled in the diagram (see \autoref{fig:task3-pre}) describes a scenario with two scenes:
Scene 1 contains a 2D Video (with ID 5), Scene 2 contains a 3D Video (with ID 4).
The scenes are initially \textbf{Hidden} and become \textbf{Visible} after the user enters in their proximity zone.
\begin{enumerate}
    \item When 3D Scene1 becomes \textbf{Visible} add, to the 2D Video, the \textit{State Tag} \textbf{Visible} and set its opacity to 90\%.
    \item To the last \textit{State Node} containing the 3D Scene 1 add two distinct \textit{Action Nodes} (respectively with the 2D Video and the 3D Scene) starting from the last created \textit{State Node} and model their final \textit{State Nodes}:
        \begin{itemize}
            \item[-]If the user \textbf{Taps} on the video it becomes \textbf{On Play}.
            \item[-]If the user exits (\textbf{Proximity Out}) from Scene 1, the 3D Scene becomes again \textbf{Hidden}, after 20 seconds, so link this \textit{Action Node} to the initial \textit{State Node} where Scene 1 is \textbf{Hidden}.
        \end{itemize}
    \item Change the \textbf{On Play} settings of the 3D Video with ID 4 to make it playing in loop.
    \item The IDs of the elements must be coherent in the whole sequence.
\end{enumerate}
The solution to this task is shown in \autoref{fig:task3-sol}.
\begin{figure}[h]
    \centering
    \includegraphics[width=\linewidth]{Figures/Evaluation/Tasks/task3-pre.png}
    \caption{Task 3}
    \label{fig:task3-pre}
\end{figure}

\begin{figure}[h]
    \centering
    \includegraphics[width=\linewidth]{Figures/Evaluation/Tasks/task3-sol.png}
    \caption{Solution to task 3}
    \label{fig:task3-sol}
\end{figure}


\subsection*{Task 4}
\textit{Instructions:} 
Given the experience already modeled in the diagram (\autoref{fig:task4-pre}):
\begin{itemize}
    \item[-] Change the name of the Canva from “New recipe” to “Task 4”
\end{itemize}
Model the following interactions in Scene 1:
\begin{itemize}
    \item[-]When a \textbf{Tap} action is performed on the Menu (with ID 5), the 3D Video ends in a state with its \textbf{Audio} set to \textbf{Off} and its \textbf{Subtitles} set to \textbf{On}.
\end{itemize}
Model the following interactions in Scene 2:
\begin{enumerate}
    \item Inside the empty \textit{Action Node}, after the 3D Model becomes \textbf{Selected}, add the action by the user that goes away (\textbf{Proximity Out}) from the 3D Model and getting closer to (\textbf{Proximity In}) the 3D Video.
    \item In the final state (from the last modeled action) the 3D Video starts \textbf{Blinking} while the 3D Model becomes \textbf{Hidden}.
    \item[*] The IDs of the elements must be coherent in the whole sequence.
\end{enumerate}
At the end, SAVE the experience “Task 4”. \\
The solution to this task is shown in \autoref{fig:task4-sol}.
\begin{figure}[h]
    \centering
    \includegraphics[width=\linewidth]{Figures/Evaluation/Tasks/task4-pre.png}
    \caption{Task 4}
    \label{fig:task4-pre}
\end{figure}

\begin{figure}[h]
    \centering
    \includegraphics[width=\linewidth]{Figures/Evaluation/Tasks/task4-sol.png}
    \caption{Solution to task 4}
    \label{fig:task4-sol}
\end{figure}
\section{Study Execution}
\label{sec:evaluation-execution}
\section{Results}
\label{sec:evaluation-results}
The analysis of results consisted in the last phase of the study; during this phase data have been manipulated and transformed into aggregated data by the authors. We analyzed each participant's screen recording to calculate the task completion rate and report this measure, in conjunction with the duration of each task, in a spreadsheet; the same operation happens for the SUS Score, automatically computed by the remote testing platform according to Brooke's formula \cite{brooke_sus_1996}, and the demographic answers. Eventually, qualitative answers are reported and correlated with quantitative results.

\subsection*{Participants}
We carried out the study with nine participants (2 women and 7 men) between 23 and 30 years old, three of which with a computer science background but only two belong to user profile P2, i.e. with experience in the AR/VR development field. We believe that, according to Nielsen et al. \cite{nielsen_mathematical_model_usability}, nine participants are enough to cover all the usability issues, where their model showed that five to eight participants can identify 80-85\% of these problems.

\subsection{Quantitative Results}
\subsubsection*{Effectiveness}
In order to measure the effectiveness through the task completion rate, we identified shorter operations per each task (\emph{sub-tasks}) and rated each participants' task by marking which of these sub-tasks has been performed, then calculating the ratio over the total number of sub-tasks. The number of operations per task are summarized as follows, while their detailed list is included in \autoref{appendix:subtasks}: Task 1 (9 sub-tasks), Task 2 (10 sub-tasks), Task 3 (10 sub-tasks), Task 4 (11 sub-tasks).

In the measurement of the effectiveness contributed only eight over nine participants, due to a technical issue that did not allow the playback of user's screen recording and its consequently analysis of the tasks. However we decided to include this participant's metrics for the next quantitative and qualitative measurements. The final effectiveness, measured as the average of task-related effectiveness (\autoref{tab:avg-effectiveness}), is 92,93\% showing that, overall, participants performed well in the study execution as also demonstrated by \autoref{fig:task-participant-effectiveness}.
\begin{table}[H]
    \centering
    \begin{tabular}{lr}
    \hline
           & \multicolumn{1}{l}{Effectiveness (Average Task Completion Rate)} \\ \hline
    Task 1 & 95,83\% (SD = 11,79\%)                                           \\
    Task 2 & 96,25\% (SD = 10,61\%)                                           \\
    Task 3 & 88,75\% (SD = 8,35\%)                                            \\
    Task 4 & 90,91\% (SD = 10,87\%)                                           \\ \hline
    \end{tabular}
    \caption{Effectiveness per task}
    \label{tab:avg-effectiveness}
\end{table}

\begin{figure}[H]
    \centering
    \includegraphics[width=\linewidth]{Figures/Evaluation/results/task-completion-rate_per-participant.png}
    \caption{Task Completion Rate per Participant}
    \label{fig:task-participant-effectiveness}
\end{figure}
Among the tasks, it resulted that the first two were completed with less problems than the third and fourth ones, in detail only participants 1 and 2 had problems, respectively, with task 2 and task 1 while all the other participants achieved 100\% of completed sub-tasks. The last two tasks, however, were completed with a slightly smaller effectiveness but still they proved that users were able also to manage more complex modelled experiences.

\subsubsection*{Efficiency}
\autoref{fig:completion-times} shows the average time required by users to complete each task with its standard deviation, resulting in an average efficiency of 398 seconds (6 minutes 38 seconds) per task per participant.
\begin{table}[h]
\centering
\begin{tabular}{lr}
\hline
       & \multicolumn{1}{l}{Efficiency (average task completion time)} \\ \hline
Task 1 & 306s (SD = 103,56s)                                           \\
Task 2 & 354s (SD = 109,40s)                                           \\
Task 3 & 522s (SD = 168,84s)                                           \\
Task 4 & 410s (SD = 116,13s)                                           \\ \hline
\end{tabular}
\caption{Efficiency per task}
\label{tab:avg-efficiency}
\end{table}

\begin{figure}[htbp]
    \centering
    \includegraphics[width=0.8\linewidth]{Figures/Evaluation/results/completion-times.png}
    \caption{Completion times per task (in seconds)}
    \label{fig:completion-times}
\end{figure}
By comparing the average efficiency values (\autoref{tab:avg-efficiency}) with the effectiveness (\autoref{tab:avg-effectiveness}) it is possible to notice that participants six and eight achieved the highest rate of effectiveness per completion time. Even though their profile (P2) sample is too small to draw rigorous conclusions, this detail might suggest that their experience in computer science and AR/VR development results in better performances in efficiency and effectiveness than non-expert users. However, further research is needed to validate this hypothesis.
\begin{figure}[htbp]
    \centering
    \includegraphics{Figures/Evaluation/results/completion-rate-time.png}
    \caption{Comparison of effectiveness and efficiency per participant}
    \label{fig:completion-rate-times}
\end{figure}

\subsubsection*{Satisfaction}
The \gls{SUS} Questionnaire obtained an average score of 75.0/100.0 with a standard deviation of 11.2\%, a minimum rating of 62.5/100.0 and a highest score of 100.0/100.0 (\autoref{fig:sus-results}).
\begin{figure}[h]
    \centering
    \includegraphics{Figures/Evaluation/results/sus.png}
    \caption{SUS Score Distribution}
    \label{fig:sus-results}
\end{figure}

The \glsxtrlong{SUS} is an effective and low-cost tool to assess the usability of a product, yet it is not intuitive how to relate the numeric 100-points scale to an absolute judgement. That being the case, basing our evaluation on the work of Bangor et al.~\cite{bangor_determining_2009} we adopted their adjective rating and acceptability scale models in \autoref{fig:bangor-sus} to provide a qualitative judgement based on quantitative results. The overall outcome of the questionnaire can then be evaluated as a "Highly Acceptable" and "Good" editor's usability as perceived from users.

\begin{figure}[h]
    \centering
    \includegraphics[width=\linewidth]{Figures/Evaluation/results/bangor-sus.png}
    \caption{A comparison of acceptability ranges, grade scale and adjective ratings with SUS Score \cite{bangor_determining_2009}}
    \label{fig:bangor-sus}
\end{figure}

\subsection{Qualitative Results}
From the qualitative open ended questionnaire we collected some users' feedback regarding the experience in using the tool. Starting from the answers regarding general problems during the use of the tool, no users found difficulties in completing the tasks but some, instead, had difficulties in dragging and dropping elements into the diagram due to the number of warning messages appearing on the top right corner of the application. These messages occur when logical rules regarding connections among elements and tags are validated, i.e. during the drag movement, resulting in potential multiple warnings. 

Other critiques regarded the few information given by the video tutorial or the ambiguity of some task descriptions; at the same time, other comments instead dissent from these opinions underlining the lack of problems faced and the conciseness of the tutorial.

The analysis of the second open ended question on positive comments about the experience showed how much participants liked using the ART Editor, defining it "Not too much complicated" (Participant 1) and "Easy to use after watching the tutorial, could be used also from people not used to this kind of tool" (Participant 2). 

Overall, then, we consider these answers as a validation of the quantitative usability test results, especially the ones on effectiveness and satisfaction, that confirmed how high scores have been achieved thanks to "The simplicity of use and the fact that there are all the things you need to build a scene in XR" (Participant 6).

Coming to the further general feedback and comments question, few users gave suggestions on possible improvements: one user pointed that sometimes, after a mistake, it is difficult to understand how to delete the last action performed (there are no 'undo' and 'redo' buttons) while another user found the application's \gls{GUI} acceptable for a prototype but less attractive if implemented for a future commercial product. 

The last constructing comment was on the role of 3D Scenes as containers of virtual elements, and the lack of clarity on what happens when elements belongs to each scene -- more specifically, it is not clear what happens if a scene is hidden but the elements in it are not. To answer this, we believe a better explanation in the tutorial on scenes and their properties should be given, in order to let other users understand that hidden scenes override the visibility property of their components.

\subsection{Findings and Next Improvements}
From the answers of the qualitative study we highlighted some improvements that could be done to the platform in order to enrich its correctness and usability: firstly the logical rules validation should happen only when elements are dropped to avoid disturbing and incoherent warning messages during user interactions; a second useful feature would be the presence of a logging tool that allows to undo and redo adjustments to the diagram. In conclusion a third feature could be a better representation of 3D Scenes as nodes more than elements as the others composing a scene; in this way it is easier to understand the concept of scenes and place the elements inside them.
Other findings have been reported while analyzing participants' screen recordings in the assessment of their effectiveness scores.

During the first task, we noted that two users didn't understand a clear distinction between state and actions, returning back several times to the task description and the tutorial to check for their definition. Only one, instead, tried to change the identifier to all the three elements regardless of their element type, resulting in the warning message "You can't set equal IDs to different object types" when trying to change the 2D Text identifier to the same of the other two 3D Models.

In the second task we noticed how, sometimes, the distinction between input and output ports in Action Nodes is not clear and should, therefore, be explained better. The last tasks showed two conceptual errors: in both the exercises two users failed in describing correctly an action first by placing two elements inside one node, resulting in their conjunction of action events rather than their disjunction, and then by modelling the opposite, i.e. a disjunction to describe a conjunction, whose effects have undefined behaviour and would be delegated on the XR Application Run-Time Engine implementation.

Finally, the sub-task output link of an Action Node to a previous State Node -- creating a loop -- has not been correctly achieved by all the participants who found other ways to represent it though, e.g. by adding a new state as the first one and linking it to the first action re-creating the loop but adding a level of redundancy. In these cases a clearer notation for parallel or exclusive actions as in BPMN could be supported, while a final diagram validation feature could minimize the number of states and actions of the \gls{FSM} and remove the redundancy issue.