\chapter{ART Editor Usability Evaluation}
\label{ch:evaluation}

The evaluation of the ART Editor aims at testing its usability according to measures defined in the standard ISO 9241-11 \cite{iso20189241} -- namely \emph{effectiveness}, \emph{efficiency} and \emph{satisfaction} -- by users to achieve the test goals. While the data analysis metrics will be described in \autoref{sec:evaluation-approach}, along with the recruitment of participants and the kind of test performed, a clear definition of these measures is given:
\begin{definition}[Effectiveness]
\label{def:effectiveness}
A measure of the accuracy and completeness with which users achieve specified goals.
\end{definition}

\begin{definition}[Efficiency]
A measure of the resources expended in relation to the accuracy and completeness with which users achieve goals.
\end{definition}

\begin{definition}[Satisfaction]
The freedom from discomfort, and positive attitudes towards the use of the product.
\end{definition}

Goal of the test is to prove that the Editor has been correctly designed to provide participants a useful and user-friendly tool to develop \gls{XR} experiences without difficulties. At the end of the test, we expect users to have accomplished the following goals:
\begin{itemize}
    \item[\textbf{G1}] Users familiarize with the interface, its menu and interactions.
    \item[\textbf{G2}] Users are able to manipulate a scene design by modeling a simple scenario.
    \item[\textbf{G3}] Users are able to design multiple scenes in the application starting from a textual description.
    \item[\textbf{G4}] Users know how to customize the high-level description of a scene, adding labels, tags and secondary information.
    \item[\textbf{G5}] Users are able to export an already designed experience.
\end{itemize}

In order to provide accurate results in accordance to the final users and stakeholders of the ART Framework, we identified two participant profiles that could fit as users in our study:
\begin{itemize}
    \item[\textbf{P1}] Users without any experience in \gls{AR}/\gls{VR} development neither in computer science. This type of users reflect the role of art curators in a touristic domain and are useful to test the easiness and level of abstraction from technical details usually involved in the development of these applications.
    \item[\textbf{P2}] Users with experience in \gls{AR}/\gls{VR} development and with a computer science background. This type of users do not reflect the end user of the framework; however, we included them not only for easiness of finding this kind of participants in a technical university, but also because they could validate the rightness of application interface abstraction and concept abstraction of the designed solution.
\end{itemize}

\section{Usability Evaluation Approach}
\label{sec:evaluation-approach}

\subsection{Usability Questionnaire}
\section{Scenarios and Tasks}
\label{sec:evaluation-scenarios}
\section{Study Execution}
\label{sec:evaluation-execution}
\section{Results}
\label{sec:evaluation-results}