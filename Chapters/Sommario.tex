\chapter{Sommario}
Al giorno d'oggi gli approcci esistenti per la creazione di esperienze in Exted Reality (XR) consistono nella maggior parte dei casi nell'utilizzo di ambienti di sviluppo integrato (IDEs) o librerie software (APIs), mentre l'adozione di un approccio ad alto livello -- che consenta di astrarre dall'implementazione -- è tuttavia limitato a metodi dipendenti dalla tecnologia o stretti discendenti di meta-modelli standard.
È necessario, quindi, un modello universale che permetta di progettare a livello concettuale applicazioni XR, senza ricorrere a soluzioni che siano strettamente dedicate al dominio o che escludano dalla progettazione i non esperti di programmazione. In questa tesi viene presentato il Modello XRM, un approccio teorico orientato alla concettualizzazione di esperienze XR, sviluppato a partire da uno studio comparativo dei modelli esistenti in letteratura. Il Modello XRM è stato adottato per l'implementazione di un Editor ad alto livello a supporto dei designer di applicazioni XR, il quale è stato validato in un test di usabilità con utenti basato sullo standard ISO 9241-11. In conclusione si è considerato come caso di studio un'esperienza progettata per un complesso archeologico allo scopo di mostrare i benefici risultanti dall'adozione dell'XRM e dell'ART Editor.


\paragraph{Parole chiave} Human-Computer Interaction; Interaction Design; Extended Reality; Conceptual Modelling; Authoring tools; Tourism; Cultural Heritage