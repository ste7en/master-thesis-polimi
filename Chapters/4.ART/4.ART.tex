\chapter{The ART Framework}
\label{ch:art}

The ART Framework is a project funded by EIT Digital in collaboration with Politecnico di Milano, Fifthingenium S.r.l.s., Technische Universität Berlin and TIM that aims at creating an Extended Reality as a Service (XRaaS) platform to enhance tourism in places with a high historical, cultural and artistic value. The platform aims to make programming experiences in XR accessible to non-IT experts. By stimulating the spread of XR applications, it is hoped that the tourism sector will also benefit from significant growth. Different stakeholders are involved, as tour operators, event organizers and museum curators, with the mission to create a unique tourist experience, starting from the decision-making process for a destination to the on-site experience, combining artistic locations with virtual tour guides and digital contents.

As we have described in \autoref{sec:background-tourism-xr}, \gls{XR} technologies are finally mature to be widely deployed in the tourism sector, also thanks to the variety of devices coming on the market. Furthermore, the new 5G internet connection will enable high-quality low-latency streaming of contents, higher download speeds, real-time interactions and multi-user experiences.

The ART project will hence provide a service for all the players involved in the tourism industry that allows the creation of \gls{XR} experiences without the need to code and deploy multiple applications but directly supporting the acquisition of digital contents, their processing and final transformation into assets downloadable and playable on different \gls{AR}/\gls{VR} devices where these experiences are rendered.

In this chapter, \autoref{sec:art-overview} will give an overview to the technological stack of ART and the workflow that enables the creation of XR apps; later, \autoref{sec:art-editor} will explain in detail the ART Editor,the application supporting the authoring process of these experiences,  inspired by the conceptual model introduced in \autoref{ch:conceptual-model}.

\section{System Overview}
\label{sec:art-overview}

The system architecture of the ART XRaaS is composed by different modules, each reflecting a phase of the creation process of an \gls{XR} experience (\autoref{fig:architecture}). The system is composed as follows:
\begin{itemize}
    \item A \textbf{Content Management System} (CMS) implements the \emph{assets upload phase}. It is a web application acting as entry point into the creation process, where designers of the \gls{XR} experience create or open an existing project and upload all the digital contents to be included in the next phases. It is served by a database to persist and retrieve assets, which can be 3D models (\textit{.fbx, .obj}), images or textures (\textit{.jpeg, .bmp, .png}), video clips (\textit{.mp4}), audio tracks (\textit{.mp3}), text documents (\textit{.txt}), QR Codes (described by their plain text content) and object-related operational entities (a \textit{menu}). Lastly, the application allows to categorize and label these assets, assigning them to different \emph{Scenes}.

    \item The \textbf{ART Editor} is involved in the \emph{interaction authoring phase}. It is a web application that, after retrieving all the assets already uploaded in the previous phase, allows to model in a canvas the interactions among these entities according to user inputs, using a "state-action-effect" paradigm typical of Finite State Machines (FSM). Indeed, the experience is described by a sequence of States connected by Actions, where each State contains different objects and each object has specific properties that can change. After modeling the interactions, the Editor allows to save the flow of the experience in a configuration file used in the next phase.
    
    \item A \textbf{\gls{HMD} Authoring Application} allows the \emph{on-site authoring phase}: during this phase the user, wearing a \gls{HMD}, opens the created project and a \emph{run-time engine} parses the configuration file exported in the previous phase; then, the application downloads all the assets from the CMS' database to proceed in the authoring process. During this phase the authors of the experience set each Scene, placing the assets in the virtual environment and configuring their properties such as position, scale and orientation; in the case of \gls{AR} authoring they map the coordinates of the physical environment with the digital one through \emph{Spacial Anchors}\footnote{\url{https://azure.microsoft.com/en-us/services/spatial-anchors/}} to persist the coordinates of digital contents with respect to real environment. 
    
    \item The on-site authoring phase is the last phase of the development process and, for complimentary, here we cite the last component of the system, that is the \textbf{Universal \gls{XR} Application Reader}. This is the final application for the end user (the tourist) that consists in a run-time engine responsible for the correct parse of the configuration file and rendering of the virtual environment. This application is available for both \glspl{HHD} and \glspl{HMD}, and in the latter case both for \gls{AR} and \gls{VR} devices.
\end{itemize}

\begin{figure}[h]
    \centering
    \includegraphics[width=\textwidth]{Figures/Editor/architecture.png}
    \caption{System Overview - Architecture}
    \label{fig:architecture}
\end{figure}

\subsection{ART User Journey}
\label{subsec:art-user-journey}

To give the reader a better understanding of the way the system works we provide a user journey considering the example of a cultural heritage site whose curators are interested in enhancing the experience of their visitors through \gls{XR} technologies. Their goals is to augment the information of the exhibits through virtual panels containing descriptions, historical videos, pictures and reconstruction models of the past; using the ART Framework it is only required some expertise in 3D content authoring to create the models, while the app development part is totally transparent to the user thanks to ART XRaaS. The user journey to develop the \gls{XR} experience would be the following:
\begin{enumerate}
    \item The user produces 3D models and gathers all the digital and multimedia assets contributing to the experience.
    \item The user accesses the web portal of the ART CMS and create a new project.
    \item The user upload all the assets on the CMS, then they define their properties such as names and labels to support the next development phase and, if necessary, cluster each of them in different scenes.
    \item The user opens the ART Editor and define the end user (tourist) interactions with the elements through a canvas and a "state – action – effect" paradigm.
    \item The user save the modelled experience and a configuration file is exported.
    \item The user, wearing a Microsoft HoloLens 2 \gls{HMD}, opens the exported configuration file using the on-site authoring application.
    \item The user decide to create an \gls{AR} application and scan the heritage site to place Spatial Anchors in it; to do so, they place the assets in the environment choosing their placement, size and orientation by manipulating these elements through \emph{gestures}.
    \item Finally, they save and publish the app that will be available for tourists, regardless of the \gls{AR} device used to augment their visit.
\end{enumerate}
\section{ART Editor}
\label{sec:art-editor}

The ART Editor is ART Framework's module involved in the interaction authoring phase. As already explained in \autoref{subsec:art-editor-from-conceptual} it is based on the XRM (\autoref{ch:conceptual-model}) and it allows to design the model of XR experiences.
\begin{figure}[h]
    \centering
    \includegraphics[width = \linewidth]{Figures/Editor/main-UI.png}
    \caption{ART Editor UI}
    \label{fig:editor-main-ui}
\end{figure}

Before precede in the next sections about the Editor it should be noted that this application is at an intermediate phase between the uploading of contents and their on-site authoring. The left side section of the (\autoref{fig:editor-main-ui}) shows a list of elements which ideally corresponds to the actual assets to be used, replaced for now by placeholders to give an overview of how they are positioned in the UI. For this reason the assets whose interaction can be modelled are, theoretically, the same uploaded on the CMS platform when a new project is created, but for testing purposes in our examples they have been replaced with elements representing their own categories (e.g. \emph{3D Model}, \emph{2D Video}, ...), given the impossibility to reproduce a working integrated CMS at the time of development.

\subsection{From the Conceptual Model to the Editor}
\label{subsec:art-editor-from-conceptual}

Basing our study on the XRM Conceptual Model (XRM) in \autoref{ch:conceptual-model} we adapted and modified it to suit a user-friendly and interactive modelling paradigm that would satisfy Fifthingenium's company needs and requirements of the component to implement.

Starting from the first part of the XRM, the Structural Model (\autoref{sec:conceptual-structural}), we decided to include as elements of the Editor only the fourth and, mostly, fifth level of abstraction (\autoref{fig:Legend}) of Non-Human Actors, as they represent a concrete category of assets that the editor can manage. Hence, the elements available in the Editor are: the \textit{QR Code} as only Interaction Placeholder element, \textit{Static 3D Model}, \textit{Text}, \textit{Image}, \textit{Dynamic 3D Model} and \textit{Video} (respectively referred to as \textit{3D Video} and \textit{2D Video}), \textit{Menu} and the \textit{360° Video} and \textit{3D Scene} as Environments. The Environemnt, differently from XRM, is not dependent on the technology but is referred as a sub-part of the Environment or a container of the objects and areas in it.
\begin{table}[h]
\centering
\begin{tabular}{|l|r|}
\hline
\textbf{XRM Actors}             & \textbf{ART Editor Elements} \\ \hline
QR Code                & QR Code             \\ \hline
Static 3D Model        & 3D Model            \\ \hline
Text                   & 2D Text             \\ \hline
Image                  & 2D Image            \\ \hline
Dynamic 3D Model       & 3D Video            \\ \hline
Video                  & 2D Video            \\ \hline
Menu                   & Menu                \\ \hline
360° Video Environment & 360° Video          \\ \hline
\begin{tabular}[c]{@{}l@{}}Augmented Environment\\  Virtual Environment\end{tabular} & 3D Scene \\ \hline
\end{tabular}
\caption{Comparison of XRM Actors and ART Editor Elements}
\label{table:xrm-art-actors}
\end{table}

Compared to the Behavioural Model (\autoref{sec:conceptual-behavioral}), in the ART Editor the Human Actor is implicitly involved in each interaction, so it hasn't been included in the modelling environment. A design choice was not to show all Actors' properties in their editor's counterpart because some of them are bounded to the nature of the digital asset being uploaded -- therefore these properties are set during the assets upload phase -- others, instead, (e.g. 3D Scene's World Anchors, Virtual Objects' position and orientation) are set during the on-site authoring phase and setting these properties at this authoring level would add an additional layer of difficulty not in the scope of the ART Project. For these reasons we chose to represent the change of elements' structural properties using a 'decoration' approach, allowing to visualize the relevant changes of properties with "tag elements" attached to each element; the same notation has been adopted with the interactions, keeping the \gls{UI} coherent and organizing elements into state and action nodes.
A comparative example is presented in \autoref{fig:comparison-effect-xrm-art}, in which the effect of making a 3D Model visible in XRM (\autoref{fig:xrm-effect}) is represented as an additional Visible "tag" in ART Editor (\autoref{fig:art-effect}) that indicates the visibility property set to on, with an opacity value of 100\%.
\begin{figure}[h]
    \begin{subfigure}{0.5\textwidth}
        \centering
        \includegraphics[width=\columnwidth]{Figures/Editor/xrm-effect.png}
        \caption{XRM Effect}
        \label{fig:xrm-effect}
    \end{subfigure}
    \begin{subfigure}{0.45\textwidth}
        \centering
        \includegraphics[width=0.5\textwidth]{Figures/Editor/art-effect.png}
        \caption{ART Editor State Tag}
        \label{fig:art-effect}
    \end{subfigure}
    \caption{Comparison of XRM Effect notation with ART Editor State Tag notation}
    \label{fig:comparison-effect-xrm-art}
\end{figure}

The concepts of Interaction and Task are derived from the Interaction Model (\autoref{sec:conceptual-interaction}): the former uses a structure very similar to the XRM (\autoref{fig:xrm-interaction}), adding user actions as "action tags" to the target (\autoref{fig:art-action}) in nodes called Action that can contain more than one interaction performed at the same time; finally, a Task in ART is implemented in the editor as a link between an Action and the corresponding effects -- namely, a State including the target elements with their appropriate tags representing a change of structural properties (\autoref{fig:comparison-task-xrm-art}).
\begin{figure}[h]
    \begin{subfigure}{0.5\textwidth}
        \centering
        \includegraphics[width=0.85\columnwidth]{Figures/Editor/xrm-interaction.png}
        \caption{XRM Interaction}
        \label{fig:xrm-interaction}
    \end{subfigure}
    \begin{subfigure}{0.45\textwidth}
        \centering
        \includegraphics[width=0.5\textwidth]{Figures/Editor/art-action.png}
        \caption{ART Editor Action}
        \label{fig:art-action}
    \end{subfigure}
    \caption{Comparison of XRM Interaction with ART Editor Action}
    \label{fig:comparison-action-xrm-art}
\end{figure}

\begin{figure}[H]
    \begin{subfigure}{\textwidth}
        \centering
        \includegraphics[width=0.7\columnwidth]{Figures/Editor/xrm-task.png}
        \caption{XRM Task}
        \label{fig:xrm-task}
    \end{subfigure}
    \begin{subfigure}{\textwidth}
        \centering
        \includegraphics[width=0.7\columnwidth]{Figures/Editor/art-task.png}
        \caption{ART Editor Task}
        \label{fig:art-task}
    \end{subfigure}
    \caption{Comparison of XRM Task with ART Editor Task}
    \label{fig:comparison-task-xrm-art}
\end{figure}
\subsection{High-Level Design}
\label{subsec:art-editor-highlevel}
\subsection{Low-Level Design}
\label{subsec:art-editor-lowlevel}

The ART Editor low-level design includes all the choices related to UI, UX and the technological implementation of the application.
From the specifications described in \autoref{subsec:art-editor-highlevel} we started defining the interaction method, consisting in a \emph{drag-and-drop} pattern used to select the objects to model and place them into the state diagram to achieve a representation as in \autoref{fig:art-fsm-sketch}; this option allows a natural and familiar user experience, already implemented by other popular visual editors like Diagrams.net\footnote{previously known as draw.io} and authoring tools like WAAT \cite{de_paolis_waat_2020}, Meta-AR \cite{villanueva_meta-ar-app_2020}, ComposAR and Wikitude Studio \cite{stephanidis_authoring_2020}.

\subsubsection*{UI/UX Design Choices}
\label{subsub:art-editor-uiux}
Some considerations have been made during the design process of the editor, mostly related to the UI to be developed. The first concern was on how to structure a state diagram for XR apps to include and represent more than one object at once when describing their state: indeed, using a \gls{FSM} model where each state is represented by a virtual object would add redundancy of actions and entry states or add inconsistencies of representations (e.g. the same input makes the \gls{FSM} go into two or more different states). For these reasons we decided to represent both states and user actions with nodes of the diagram describing states and inputs of the \gls{XR} finite-state machine; these nodes act as containers of elements -- dragged and dropped into them -- where their behaviour and their properties are delineated. Thus, one or more virtual objects can be a member of state (\autoref{fig:stateNode}) and action nodes (\autoref{fig:actionNode}), with links among them acting as links between states in a standard state diagram.

\begin{figure}[h]
    \centering
    \includegraphics[width=\textwidth]{Figures/Editor/wireframes/stateNode.png}
    \caption{Low Level Design - State Node}
    \label{fig:stateNode}
\end{figure}

\begin{figure}[h]
    \centering
    \includegraphics[width=\textwidth]{Figures/Editor/wireframes/actionNode.png}
    \caption{Low Level Design - Action Node}
    \label{fig:actionNode}
\end{figure}

Another important aspect to consider was the definition of user actions on elements and their effects as changes of structural properties. Basing our decisions on the drag-and-drop interaction pattern and the clustering of states and actions (\autoref{subsec:art-editor-highlevel}) we introduced two complementary elements able to describe elements' states and actions inside their nodes: \emph{state} and \emph{action tags}. These components (\autoref{fig:tags}) designate atomic properties or interactions to be dragged and dropped to compose states and actions (\autoref{fig:sasTags}); in this way it is possible to design XR applications modelling only the required elements and their properties, allowing an easy and understandable state diagram (\autoref{fig:sasLink}) of the entire app life-cycle.

\begin{figure}[h]
    \centering
    \includegraphics[width=\textwidth]{Figures/Editor/wireframes/tags.png}
    \caption{Low Level Design - Tags}
    \label{fig:tags}
\end{figure}

\begin{figure}[h]
    \centering
    \includegraphics[width=\textwidth]{Figures/Editor/wireframes/SaSTags.png}
    \caption{Low Level Design - State and Action Tags}
    \label{fig:sasTags}
\end{figure}

\begin{figure}[h]
    \centering
    \includegraphics[width=\textwidth]{Figures/Editor/wireframes/SaSlinks.png}
    \caption{Low Level Design - State and Action Link}
    \label{fig:sasLink}
\end{figure}

After state tags have been attached to an element it is possible to define the structural properties of these (\autoref{tab:state-tag-properties}), belonging to the Non-Human Actors' Structural Model properties (\autoref{sec:conceptual-structural}) or interaction properties (\autoref{tab:action-tag-properties}) for specific actions.

\begin{table}[h]
\resizebox{\textwidth}{!}{%
\begin{tabular}{|l|l|l|}
\hline
\textbf{State}           & \textbf{Property}         & \textbf{Description}                                                                               \\ \hline
\multirow{4}{*}{Visible} & opacity =  {[}0-100{]}\%  & The opacity of the visible element                                                                 \\ \cline{2-3} 
                             & animate = \{true | false\}   & \begin{tabular}[c]{@{}l@{}}Choose whether the element should \\ become visible fading in or not\end{tabular} \\ \cline{2-3} 
                         & duration {[}s{]}          & \begin{tabular}[c]{@{}l@{}}Duration of the fade in animation \\ (in seconds)\end{tabular}          \\ \cline{2-3} 
                         & wait {[}s{]}              & \begin{tabular}[c]{@{}l@{}}Waiting time before starting the animation \\ (in seconds)\end{tabular} \\ \hline
\multirow{3}{*}{Hidden}      & animate = \{true | false\}   & \begin{tabular}[c]{@{}l@{}}Choose whether the element should \\ become hidden fading out or not\end{tabular} \\ \cline{2-3} 
                         & duration {[}s{]}          & \begin{tabular}[c]{@{}l@{}}Duration of the fade out animation \\ (in seconds)\end{tabular}         \\ \cline{2-3} 
                         & wait {[}s{]}              & \begin{tabular}[c]{@{}l@{}}Waiting time before starting the animation \\ (in seconds)\end{tabular} \\ \hline
Selected                     & highlight = \{true | false\} & \begin{tabular}[c]{@{}l@{}}Highlight the selected element with \\ luminous borders\end{tabular}              \\ \hline
\multirow{3}{*}{Manipulated} & rotation = \{true | false\}  & \begin{tabular}[c]{@{}l@{}}Choose whether to allow the element \\ to be rotated or not\end{tabular}          \\ \cline{2-3} 
                         & move = \{true | false\}   & \begin{tabular}[c]{@{}l@{}}Choose whether to allow the element \\ to be moved or not\end{tabular}  \\ \cline{2-3} 
                         & scale = \{true | false\}  & \begin{tabular}[c]{@{}l@{}}Choose whether to allow the element \\ to be scaled or not\end{tabular} \\ \hline
Blinking                 & loop = \{true | false\}   & \begin{tabular}[c]{@{}l@{}}Choose whether to blink the element \\ in loop or not\end{tabular}      \\ \hline
On Play                  & loop = \{true | false\}   & \begin{tabular}[c]{@{}l@{}}Choose whether to restart the playback \\ at end or not\end{tabular}    \\ \hline
Audio                    & sound = \{on | off\}      & Control the sound of multimedia elements                                                           \\ \hline
Subtitles                & visibility = \{on | off\} & Control the subtitles of multimedia elements                                                       \\ \hline
\end{tabular}%
}
\caption{Structural properties of state tags}
\label{tab:state-tag-properties}
\end{table}

\begin{table}[h]
\resizebox{\textwidth}{!}{%
\begin{tabular}{|l|l|l|}
\hline
\textbf{Action} & \textbf{Property} & \textbf{Description}                             \\ \hline
Proximity In  & duration {[}s{]} & \begin{tabular}[c]{@{}l@{}}The amount of elapsed time into an element's\\ proximity area\end{tabular}         \\ \hline
Proximity Out & duration {[}s{]} & \begin{tabular}[c]{@{}l@{}}The amount of elapsed time out from an element's\\ proximity area\end{tabular}     \\ \hline
Gaze In         & duration {[}s{]}  & The amount of time the user looks at the element \\ \hline
Gaze Out      & duration {[}s{]} & \begin{tabular}[c]{@{}l@{}}The amount of time after the user avert their gaze\\ from the element\end{tabular} \\ \hline
\end{tabular}%
}
\caption{Properties of action tags}
\label{tab:action-tag-properties}
\end{table}

In conclusion, the last step consisted in drawing the wireframes describing the \gls{GUI} of ART Editor (\autoref{fig:canva1}, \autoref{fig:canva2}), including a left pane with the available elements to choose for modelling the experience, a main paned window containing the modelled \gls{FSM} diagram and a horizontal tab bar on top, displaying the name of the opened project and other functional buttons.

\begin{figure}[h]
    \centering
    \includegraphics[width=\textwidth]{Figures/Editor/wireframes/canva1.png}
    \caption{Low Level Design - Editor land page}
    \label{fig:canva1}
\end{figure}

\begin{figure}[h]
    \centering
    \includegraphics[width=\textwidth]{Figures/Editor/wireframes/canva2.png}
    \caption{Low Level Design - Editor Options}
    \label{fig:canva2}
\end{figure}

\subsubsection*{Technological Choices}
\label{subsub:art-editor-technological}
From the requirements and decisions made during the low-level and high-level design phases we chose to implement the diagram modelling feature using the GoJS\footnote{\url{https://gojs.net/}} web framework. GoJS is a JavaScript library that allows to build interactive diagrams and applications to model entities with custom rules and designs.

Adapting GoJS into a new project is a simple task and it only requires to include its .~js library file into the HTML page as a script tag source, while the whole implementation of the diagram is delegated to JavaScript functions executed after the web page has loaded.

Finally, our focus has shifted on the last authoring step that allows the ART Editor to export the modelled experience to open in with the \gls{HMD} on-site authoring application. Being the editor and its data model developed in JavaScript and given the compatibility and modifiability constraints of the configuration file to be exported, our choice has been to save the diagram using \gls{JSON} as data interchange technology and file format. This choice allows an easy transformation of GoJS data, i.e. the modelled experience, into a specific application readable schema (\autoref{main-json-schema}) allowing an easy integration and portability for both the applications.

\lstset{numbers=left, numberstyle=\tiny, numbersep=5pt, captionpos=b, basicstyle=\ttfamily, 
emph={ARTRecipeID,Name,Description,ListOfObjects,type,id,ListOfStates,InitState,EndState,Objects,decorations,options,ListOfTransitions,Inputs,NextState}, emphstyle=\textbf}
\begin{lstlisting}[firstnumber=1, caption={ART Editor JSON Schema}, label=main-json-schema]
{
  "ARTRecipeID": <uuid>
  "Name": <string>
  "Description": <string>
  "ListOfObjects": [
    {
      "type": <string>
      "id": <integer>
    }
  ],
  "ListOfStates": [
    {
      "Name": <string>
      "InitState": <boolean>
      "EndState": <boolean>
      "Objects": [
        {
          "type": <string>
          "id": <integer>
          "decorations": [
            {
              "type": <string>
              "options": {}
            }
          ]
        },
      ],
      "ListOfTransitions": [
        {
          "Name":  <string>
          "Inputs": [
            {
              "type": <string>
              "id": <integer>
              "decorations": [
                {
                  "type": <string>
                  "options": {}
                }
              ]
            }
          ],
          "NextState": <string>
        }
      ]
    },
  ]
}
\end{lstlisting}

As it is possible to notice from the \gls{JSON} schema in \autoref{main-json-schema} it represents a detailed view of the modelled diagram, of which three main lists constitute the core of the diagram: the \lstinline!ListOfObjects! is the list of each element participating in the scene and identified by their unique identifier, \lstinline!ListOfStates! describes each state node with a unique identifier (Name), two booleans to indicate whether it is an initial state, a final state or neither and the list of objects present in the state. The latter lists objects in a state by their id and type (3d\_scene, 3d\_model, 3d\_video, 2d\_video, 2d\_image, 2d\_text, 360\_video, menu, qr\_code, device), while the \lstinline!decorations! attribute contains a collection of state tags with their related options. In conclusion, the \lstinline!ListOfTransitions! attribute of each state designates action nodes with a unique identifier (Name), a list of objects containing the object identifiers, their action tags and related properties (decorations), and lastly a \lstinline!NextState! attribute referencing the final state of that specific action.

To give a further explanation of the JSON model we provide an example in the Appendix section (\autoref{json-schema-example}), describing a 2D Video visible and in pause that, when a tap action occurs, it plays its content in loop.
\subsection{Implementation}
\label{subsec:art-editor-implementation}
\subsection{Example of use of ART Editor: The NURE Use Case}
\label{subsec:art-editor-nure-example}

In the previous \autoref{ch:conceptual-model}, we applied the XRM Conceptual Model to a real case study provided by the company Fifthingenium. Since the model proved to be very effective in describing the experience, we decided to represent NURE also by using the ART Editor. 
As in \autoref{sec:conceptual-nure-example}, we are only modelling an extract of the inner journey for obvious reasons of space. 
Consider also that between the case study modelled with the XRM and the one using the Editor there are some semantic differences for which we refer to \autoref{subsec:art-editor-from-conceptual}.

\subsection*{Elements}

Below are the elements that are part of the excerpt from the NURE experience associated with the corresponding default elements of the ART Editor.

\begin{itemize}
    \item 3D Scene: The 3D Scene element has been used to distinguish the different scenes of the whole experience. Each scene corresponds to an Augmented Environment in the XRM Model. Respectively, AE 2 corresponds to "Scene 1", AE 2.2 to "Scene 2" and AE 3 to "Scene 3". In addition, "Scene 1" is also inserted here, since the ART Editor is based on the concepts of State-Action-Effect and therefore requires an initial state.
    \item 3D Object: The 3D Object element represents the Static Objects of the XRM Model, exactly the Tomb D, the Tomb E, the respective covering slabs, the interior and the exterior of the Mausoleum.
    \item 3D Video: The 3D Video element, corresponding to a Dynamic Object in the XRM Model, represents the reproducible scene of the ancient Refrigerium ritual.
    \item QR Code: The QR Code element corresponds to an Interaction Placeholder in the XRM Model and provides the user with an alternative to trigger a new scene.
    \item Menu: The Menu element is represented by a Button, an Operational Object, in the XRM Model and provides the user with an alternative to trigger a new scene.
\end{itemize}

\subsection*{Use Case Scenario}

Once again, it was decided to explain in detail the modelling built using the ART Editor through the use of a use case scenario, this time adopting the point of view of a user, an art curator and designer, who uses our Editor. 

Laura is 31 years old, studied Economics and Management of Arts and Cultural Activities at Ca' Foscari in Venice and now she works in an agency that deals with the design, curatorship and exhibition of cultural and artistic events. Laura received a proposal from the cultural organisation NURE which asked her to design an extended reality experience in the archaeological area of Santa Lucia di Assolo (OR) in Sardinia.

Although the idea is very exciting, Laura fears she does not have the right skills to use XR frameworks or to interact with the engineering team. After learning about the existence of the ART Editor, Laura starts to imagine the NURE experience, so she decides to compose it. After receiving the requirements and the media provided by the archaeological area authority, she can launch the Editor, which presents on the right the list of elements needed to build the experience. 

Laura decides to design the most interesting area of the whole path (\autoref{fig:nure1}, \autoref{fig:nure2}) , the one that starts from Scene 2, therefore she provides the user with a double alternative to activate Scene 2, that is by tapping on Menu2 or simply letting the user move away from Scene 1 to get closer to Scene 2.

In both cases all components of scene 2, such as Tomb D, Tomb E, the respective covering slabs, the outside of the Mausoleum are visible, except for the inside of the Mausoleum. Laura wants the 3D models to be appealing, to stimulate the visitor to interact with them, so she inserts as State Tag the Blinking, a luminous outline for each model. 
As the models are placed at progressive distances, Laura wants to give the visitor the possibility to be free to choose what to interact with first. She knows that XR experiences are dynamic experiences, where the user is encouraged to continue at a certain pace in the designed path, for several reasons including the limited battery life of the Visors. 

Moreover, she decides to insert the Swipe as an Action Tag, so that the user can slide with the finger on the slabs of the tombs to see the inside, since the covers have become transparent. Once past the tombs, Laura wants the user to see the centrepiece of this Scene, in fact when the user moves away, the tombs become hidden and through the menu the user can activate Scene 2.2, which also includes the Interior of the Mausoleum. 

Laura, has left the outside of the Mausoleum still visible and blinking but with 50\% opacity, allowing a glimpse of something inside. The user is thus encouraged to tap on the outside of the Mausoleum, to see the inside. 

Our designer decides to start the 3D video of the Refrigerium scene at the end of a 5-second timer. Once the video is over, the user can activate scene 3, again with a double option, by framing the QR code or using Menu3 of the application. 

Laura is very satisfied with her design, especially because it was easy and very intuitive to use the ART Editor. Now all she has to do is generate the JSON file and and submit it to the HMD application to complete the development of the experience.

\begin{figure}[h]
    \centering
    \includegraphics[width=\linewidth]{Figures/Editor/nure/NureExample1.png}
    \caption{ART Case Study 1st part - NURE}
    \label{fig:nure1}
\end{figure}

\begin{figure}[h]
    \centering
    \includegraphics[width=\linewidth]{Figures/Editor/nure/NureExample2.png}
    \caption{ART Case Study 2nd part - NURE}
    \label{fig:nure2}
\end{figure}